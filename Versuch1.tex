\section{Elektrochemische Spannungsreihe}
\subsection{Aufgabenstellung}
Es sollten Redoxpaare untersucht und in die elektrochemische Spannungsreihe eingeordnet werden.
\subsection{Versuchsdurchführung}
\subsubsection{Teil a}
Zuerst wurde die verdünnte Salzsäure hergestellt. Diese wurde in jeweils vier Reagenzgläser gegeben. In das erste wurde ein Stück Magnesium-Band, in das zweite
eine Zink-Granalie, in das dritte eine Spatelspitze Nickel-Pulver und in das vierte
einige Kupferspäne gegeben.
\subsubsection{Teil b}
Zuerst wurde die verdünnte Salpetersäure hergestellt. Anschließend wurde in die
Reagenzgläser je eine Zinkgranalie gegeben. In das erste Reagenzglas wurden wenige Tropfen konzentrierte Salzsäure, in das zweite wurden wenige Tropfen der
verdünnten Salpetersäure und in das dritte wurden wenige Tropfen konzentrierter
Salpetersäure gegeben. Anschließend wurden drei Reagenzgläser mit Kupferspänen befüllt. In das erste wurde konzentrierte Salzsäure, in das zweite verdünnte
Salpetersäure und in das dritte konzentrierte Salpetersäure gegeben.
\subsection{Beobachtungen}
\subsubsection{Teil a}


\subsubsection{Teil b}

\subsection{Auswertung}
\subsubsection{Teil a}
Da bei Zink, Nickel und Magnesium die gleichen Reaktionen ablaufen, wird eine
allgemeine Reaktionsgleichung mit $M$ für das Metall formuliert.

\begin{align}
    \text{Oxidation:} & \quad \ce{\overset{\pm0}{M}_{(s)} -> \overset{+2}{M}^{2+}_{(aq)} + 2e^{-}}   \\
    \text{Reduktion:} & \quad \ce{2\overset{+1}{\ce{H+}}_{(aq)} + 2e^{-} -> \overset{\pm0}{\ce{H2}}_{(g)} ^}   \\
    \rule[0.5ex]{0.12\textwidth}{0.4pt} & \rule[0.5ex]{0.55\textwidth}{0.4pt}\notag\\
    \text{Gesamt:} & \quad \ce{$M$_{(s)} + 2H^+_{(aq)} -> $M$^{2+}_{(aq)} + H2_{(g)} ^} 
\end{align}
In der Reaktion wird das Metall oxidiert und die Wasserstoff-Ionen werden reduziert. Es entsteht molekularer Wasserstoff.
\subsubsection{Teil b}
Bei der Reaktion von Zink mit der konzentrierten Salzsäure findet die gleiche
Reaktion wie in der Gleichung 6 statt.

%%%%%%%%%%%%%%%%%%%%%%%%%%Absatz dann folgt Absatz mit Korrektur%%%%%%%%%%%%%%%%%%%%%%%%%%