\documentclass[12pt]{article}
\usepackage[version=4]{mhchem}
\usepackage{amsmath}
\usepackage{amssymb}
\usepackage{geometry}
\usepackage{hyperref}
\usepackage{float}
\usepackage[ngerman]{babel}
\usepackage{array}
\usepackage{color}
\usepackage{graphicx}
\usepackage{chemmacros}
\renewcommand{\familydefault}{\sfdefault} % Serifenlose Schrift

%\fontfamily{qcr}

%preferences
\newgeometry{vmargin={30mm}, hmargin={30mm}}   % set the margins
%opening

\title{Versuch 7\\ Redoxreaktionen}
\date{}
\author{}

\begin{document}
\newcommand{\supercite}[1]{\textsuperscript{\cite{#1}}}

\begin{titlepage}
    \begin{center}
        \Large\textbf{Versuch 7 \\Redoxreaktionen}

        {\color{white}0}\\

        \small\textbf{Integriertes Praktikum}    
    \end{center}


    \begin{table}[H]
        \centering
        \begin{tabular}{>{\bfseries}l>{\bfseries}l}
            Gruppennummer: & 21a\\
            Verfasser 1: & Linus Hesmert\\
            Email: & st193339@stud.uni-stuttgart.de\\
            Matrikelnummer: & 3795053\\
            Verfasser 2: & Annika Sittlinger\\
            Email: & st192398@stud.uni-stuttgart.de\\
            Matrikelnummer: & 3780657\\
            Assistent: & Katja Engel\\
            Versuchdatum: & 26.11.2024\\
            Erstabgabe: & 29.11.2024\\
            Zweitabgabe: & 11.12.2024\\
            Drittabgabe: & leer \\
        \end{tabular}
    \end{table}

\end{titlepage}





\tableofcontents
\newpage
\section{Theorie}
Redoxreaktionen sind Reaktionen, bei denen Elektronen zwischen den Edukten
übertragen werden. Redoxreaktionen lassen sich in zwei Teilreaktionen unterteilen.
Die Oxidation ist die Teilreaktion, bei der ein Atom Elektronen abgibt. Die Reduktion ist die Teilreaktion, bei der ein Atom Elektronen aufnimmt. Diese Teilreaktionen treten immer gemeinsam auf, das bedeutet wenn eine Reduktion stattfindet
muss auch die dazugehörige Oxidation stattfinden. Ein Stoff, der selbst reduziert
wird, oxidiert andere und heißt somit Oxidationsmittel. Dies gilt umgekehrt für das
Reduktionsmittel. Es gibt zwei besondere Redoxreaktionen. Die sogenannte Synproportionierung (auch Komproportionierung) und die Disproportionierung. Bei
der Synproportionierung reagiert ein Atom welches in beiden Edukten vorkommt
zu einem Produkt. Hierbei ist zu beachten, dass das Atom des einen Edukt eine
niedrige Oxidationsstufe hat und das Atom im anderen Edukt eine hohe Oxidationsstufe hat. Im entstehenden Produkt erhält das Atom eine mittlere Oxidationsstufe. Bei einer Disproportionierung reagiert ein Edukt zu zwei Produkten, die
das gleiche Atom enthalten. Dieses Atom erhält im ersten Produkt eine niedrige
Oxidationsstufe im anderen Produkt eine hohe Oxidationsstufe. Das Atom hat im
Edukt eine mittlere Oxidationsstufe.\supercite{Jander*Blasius}

In der elektrochemischen Spannungsreihe sind Redoxpaare mit ihrem Standardpotential in Bezug zur Standardwasserstoffelektrode aufgelistet. Ein Standardpotential entspricht der Zellspannung in Bezug auf die Standardwasserstoffelektrode
als zweite Halbzelle und nur unter Standardbedingungen (Druck $p=1$ $\mathrm{atm}$, Temperatur $T=298,15$ $\mathrm{K}$, Konzentration $c=1$ $\mathrm{mol/L}$). Als Redoxpaare werden
Atome in ihrer oxidierten und zugehörigen reduzierten Form zusammengefasst.
Reagieren zwei Stoffe in einer Redoxreaktion miteinander, ist das mit dem größeren Standardpotential das,
welches als Oxidationsmittel wirkt. Das andere wirkt somit als Reduktionsmittel.
Die Standardwasserstoffelektrode beruht auf dem Redoxpaar
\begin{equation}
    \ce{H2 <=> 2H^+ + 2e-}
\end{equation}
Daraus folgt, dass alle Metalle mit einem Potential kleiner dem des Wasserstoffs
(0,0 $\mathrm{V}$) mit Säuren eine Reaktion eingehen und so die Protonen reduziert werden.
Die Metalle werden oxidiert.\supercite{skript}\\

\noindent\textbf{Nernst'sche Gleichung}\\
Die in der elektrochemischen Spannungsreihe aufgeführten Standardpotentiale $E^\theta$
gelten nur für Standardbedingungen. Sobald von diesem Standardbedingungen abgewichen wird, muss das nun geltende Elektrodenpotential mit der Nernst’schen
Gleichung berechnet werden.
\begin{equation}
    E=E^\theta + \frac{\mathrm{R}\cdot T}{z\cdot \mathrm{F}}\cdot \ln \left(\frac{c(\mathrm{ox})}{c(\mathrm{red})}\right)
\end{equation}
$c(\mathrm{ox})$ und $c(\mathrm{red})$ in der Gleichung beschreiben die Konzentration der oxidierten
oder reduzierten Form, R ist die allgemeine Gaskonstante, $T$ die Temperatur in Kelvin, $z$ die von der einen Halbzelle auf die andere Halbzelle
übertragenen Elektronen und F die Faraday-Konstante. Die Nernst’sche Gleichung beschreibt nur eine Halbzelle und nicht die gesamte
Zellspannung eines Elements.\supercite{skript}\\

\noindent\textbf{Potentialdifferenz und spontan ablaufende Reaktionen}\\
Um die Zellspannung eines Elements bei Standardbedingungen zu berechnen, werden die Standardpotentiale voneinander subtrahiert. Es gilt:
\begin{equation}
    U=E^\theta_\mathrm{Reduktion}-E^\theta_\mathrm{Oxidation}
\end{equation}
Bei Abweichungen von den Standardbedingungen werden zuerst die Standardpotentiale, in der oben stehenden Gleichung, mit der Nernst’schen Gleichung auf die
gegebenen Bedingungen "korrigiert". Ist diese Potentialdifferenz kleiner Null, so
läuft die Reaktion nicht spontan ohne Zugabe von Energie ab. Die Zellspannung
eines Elements ist zu Beginn gleich der Elektromotorischen Kraft.\supercite{skript}
\section{Elektrochemische Spannungsreihe}
\subsection{Aufgabenstellung}
Es sollten Redoxpaare untersucht und in die elektrochemische Spannungsreihe eingeordnet werden.
\subsection{Versuchsdurchführung}
\subsubsection{Teil a}
Zuerst wurde die verdünnte Salzsäure hergestellt. Diese wurde in jeweils vier Reagenzgläser gegeben. In das erste wurde ein Stück Magnesium-Band, in das zweite
eine Zink-Granalie, in das dritte eine Spatelspitze Nickel-Pulver und in das vierte
einige Kupferspäne gegeben.
\subsubsection{Teil b}
Zuerst wurde die verdünnte Salpetersäure hergestellt. Anschließend wurde in die
Reagenzgläser je eine Zinkgranalie gegeben. In das erste Reagenzglas wurden wenige Tropfen konzentrierte Salzsäure, in das zweite wurden wenige Tropfen der
verdünnten Salpetersäure und in das dritte wurden wenige Tropfen konzentrierter
Salpetersäure gegeben. Anschließend wurden drei Reagenzgläser mit Kupferspänen befüllt. In das erste wurde konzentrierte Salzsäure, in das zweite verdünnte
Salpetersäure und in das dritte konzentrierte Salpetersäure gegeben.
\subsection{Beobachtungen}
\subsubsection{Teil a}


\subsubsection{Teil b}

\subsection{Auswertung}
\subsubsection{Teil a}
Da bei Zink, Nickel und Magnesium die gleichen Reaktionen ablaufen, wird eine
allgemeine Reaktionsgleichung mit $M$ für das Metall formuliert.

\begin{align}
    \text{Oxidation:} & \quad \ce{\overset{\pm0}{M}_{(s)} -> \overset{+2}{M}^{2+}_{(aq)} + 2e^{-}}   \\
    \text{Reduktion:} & \quad \ce{2\overset{+1}{\ce{H+}}_{(aq)} + 2e^{-} -> \overset{\pm0}{\ce{H2}}_{(g)} ^}   \\
    \rule[0.5ex]{0.12\textwidth}{0.4pt} & \rule[0.5ex]{0.55\textwidth}{0.4pt}\notag\\
    \text{Gesamt:} & \quad \ce{$M$_{(s)} + 2H^+_{(aq)} -> $M$^{2+}_{(aq)} + H2_{(g)} ^} 
\end{align}
In der Reaktion wird das Metall oxidiert und die Wasserstoff-Ionen werden reduziert. Es entsteht molekularer Wasserstoff.
\subsubsection{Teil b}
Bei der Reaktion von Zink mit der konzentrierten Salzsäure findet die gleiche
Reaktion wie in der Gleichung 6 statt.

%%%%%%%%%%%%%%%%%%%%%%%%%%Absatz dann folgt Absatz mit Korrektur%%%%%%%%%%%%%%%%%%%%%%%%%%
\include{Versuch2.tex}
\include{Versuch3.tex}

\begin{thebibliography}{9}
	\bibitem{Jander*Blasius}
	Jander,
	Blasius
	\textit{Einführung in das anorganisch-chemische Praktikum},
	S.Hirzel Verlag Stuttgart, 1990.
	
    
	
	\bibitem{Löslichkeitsprodukte}
	H.Lohninger
	\textit{Tabelle von Löslichkeitsprodukten},
	\url{http://anorganik.chemie.vias.org/loeslichkeitsprodukt_tabelle.html}, Zuletzt aufgerufen am 25.12.2024



    \bibitem{skript}
    Prof. Dr. Ingo Hartenbach
    \textit{Laborpraktische Übungen\\
    -Einführung in die Chemie},
    2024
\end{thebibliography}

\end{document}
